% Don't touch this %%%%%%%%%%%%%%%%%%%%%%%%%%%%%%%%%%%%%%%%%%%
\documentclass[11pt]{article}
\usepackage{fullpage}
\usepackage[left=1in,top=1in,right=1in,bottom=1in,headheight=3ex,headsep=3ex]{geometry}
\usepackage{graphicx}
\usepackage{float}
\usepackage{longtable}

\newcommand{\blankline}{\quad\pagebreak[2]}
%%%%%%%%%%%%%%%%%%%%%%%%%%%%%%%%%%%%%%%%%%%%%%%%%%%%%%%%%%%%%%

% Modify Course title, instructor name, semester here %%%%%%%%
\title{CSC112: Introduction to Computer Science II}
\author{Dr. Robert Lowe}
\date{Spring, 2020}


% Don't touch this %%%%%%%%%%%%%%%%%%%%%%%%%%%%%%%%%%%%%%%%%%%
\usepackage[sc]{mathpazo}
\linespread{1.05} % Palatino needs more leading (space between lines)
\usepackage[T1]{fontenc}
\usepackage[mmddyyyy]{datetime}% http://ctan.org/pkg/datetime
\usepackage{advdate}% http://ctan.org/pkg/advdate
\newdateformat{syldate}{\twodigit{\THEMONTH}/\twodigit{\THEDAY}}
\newsavebox{\MONDAY}\savebox{\MONDAY}{Mon}% Mon
\newcommand{\week}[1]{%
%  \cleardate{mydate}% Clear date
% \newdate{mydate}{\the\day}{\the\month}{\the\year}% Store date
  \paragraph*{\kern-2ex\quad #1, \syldate{\today} - \AdvanceDate[4]\syldate{\today}:}% Set heading  \quad #1
%  \setbox1=\hbox{\shortdayofweekname{\getdateday{mydate}}{\getdatemonth{mydate}}{\getdateyear{mydate}}}%
  \ifdim\wd1=\wd\MONDAY
    \AdvanceDate[7]
  \else
    \AdvanceDate[7]
  \fi%
}
\usepackage{setspace}
\usepackage{multicol}
%\usepackage{indentfirst}
\usepackage{fancyhdr,lastpage}
\usepackage{url}
\pagestyle{fancy}
\usepackage{hyperref}
\usepackage{lastpage}
\usepackage{amsmath}
\usepackage{layout}

\lhead{}
\chead{}
%%%%%%%%%%%%%%%%%%%%%%%%%%%%%%%%%%%%%%%%%%%%%%%%%%%%%%%%%%%%%%

% Modify header here %%%%%%%%%%%%%%%%%%%%%%%%%%%%%%%%%%%%%%%%%
\rhead{\footnotesize CSC1120-01 Spring 2020}

%%%%%%%%%%%%%%%%%%%%%%%%%%%%%%%%%%%%%%%%%%%%%%%%%%%%%%%%%%%%%%
% Don't touch this %%%%%%%%%%%%%%%%%%%%%%%%%%%%%%%%%%%%%%%%%%%
\lfoot{}
\cfoot{\small \thepage/\pageref*{LastPage}}
\rfoot{}

\usepackage{array, xcolor}
\usepackage{color,hyperref}
\definecolor{clemsonorange}{HTML}{EA6A20}
\hypersetup{colorlinks,breaklinks,linkcolor=clemsonorange,urlcolor=clemsonorange,anchorcolor=clemsonorange,citecolor=black}

\begin{document}

\maketitle

\blankline

\begin{tabular*}{.93\textwidth}{@{\extracolsep{\fill}}lr}
%%%%%%%%%%%%%%%%%%%%%%%%%%%%%%%%%%%%%%%%%%%%%%%%%%%%%%%%%%%%%%

% Modify information %%%%%%%%%%%%%%%%%%%%%%%%%%%%%%%%%%%%%%%%%
E-mail: \texttt{robert.lowe@maryvillecollege.edu} & Office Phone: 865-981-8169 \\

 Office Hours: MWF 1:00PM -- 2:00PM, TR 3:00PM -- 4:00PM  &  Class Hours: MW 2:00 -- 4:00\\
 Office: SSC 214 & Class Room: SSC 204\\
\hline
\end{tabular*}

\vspace{5 mm}

% First Section %%%%%%%%%%%%%%%%%%%%%%%%%%%%%%%%%%%%%%%%%%%%
\section*{Course Description}
A continuation of Computer Science 111 with emphasis on advanced
programming features. Laboratory work supplements and expands lecture
topics and offers supervised practice using programming. 


% Second Section %%%%%%%%%%%%%%%%%%%%%%%%%%%%%%%%%%%%%%%%%%%
\section*{Required Materials}
\begin{itemize}
    \item {\em Big C++. 3/e}. Cay Horstmann.
    https://tinyurl.com/bigcpp
    \item An internet connected computer of some sort.
\end{itemize}

% Third Section %%%%%%%%%%%%%%%%%%%%%%%%%%%%%%%%%%%%%%%%%%%%
\section*{Prerequisites}
CSC111


% Fourth Section %%%%%%%%%%%%%%%%%%%%%%%%%%%%%%%%%%%%%%%%%%%
\section*{Course Goals}
\begin{itemize}
    \item Gain an advanced understanding of Object Oriented Programming.
    \item Learn to use Object Oriented Analysis and Design to build large
        complex programs.
    \item Gain a preliminary understanding of low level programming
        concepts, especially memory addressing.
    \item Increase your knowledge of using and programming the
        UNIX operating system.
    \item Hone your knowledge of programming tools like
        make, g++, and gdb.
    \item Learn advanced tool usage for programs like git and gdb.  
    \item Begin to establish an online portfolio to make yourself attractive 
        to future employers.
\end{itemize}

% Fifth Section %%%%%%%%%%%%%%%%%%%%%%%%%%%%%%%%%%%%%%%%%%%
\section*{Course Structure}
\subsection*{Methods of Instruction}
\begin{itemize}
    \item Lecture Integrated with Hands-On Lab Activities
    \item Independent Programming Assignments
    \item Term Project
    \item Exams
\end{itemize}

\subsection*{Grading}
This course is graded using a weighted average among four categories.
The assignments within each category are equally weighted and are all
graded out of 100 points.  Hence your final numeric grade is
computed by finding the average of each category, and then multiplying
them by the corresponding weight.  The weights for each category are
as follows:

\begin{tabular}{|l|r|}
\hline
{\bf Category} & {\bf Weight}\\
\hline
Exams & 20\%\\
\hline
Attendance \& Hands-On Lab Activities & 20\% \\
\hline
Programming Assignments & 35\%\\
\hline
Term Project* & 25\% \\
\hline
\end{tabular}

{\footnotesize * Note that your course letter grade cannot be higher
than the letter grade you earn for the term project.}

\vspace{0.25in}

Grade reports will be returned to you via a private github
repository.  If you notice any inaccuracy in your grading, please
report it as soon as possible.  

\vspace{0.25in}

Letter grades will be assigned according to the following scale:

\begin{tabular}{|lr|lr|lr|lr|lr|}
    \hline
    A+ & 96.7--100\% & B+ & 86.7--90\% & C+ & 76.7--80\% & D+ & 66.7--70\% & F & less than 60 \% \\
    A  & 93.3--96.7\% & B & 83.3--86.7\% & C & 73.3--76.7\% & D & 63.3--66.7\% & & \\
    A- & 90--93.3\% & B- & 80--83.3\% & C- & 70--73.3\% & D- & 60--63.3\% & & \\
    \hline
\end{tabular}

\subsection*{Assessments}
The standards of assessment in each grading category will be as
follows.

\subsubsection*{Exams {\em (20\% of the final grade)}}
There will be two exams given in this class: a midterm exam and final
exam.  The final exam is not comprehensive, it merely covers the
material from the second half of the course. Both exams are mixed
format exams including multiple choice, fill-in-the-blank, and short
answer questions.

\subsubsection*{Attendance \& Hands-On Lab Activities {\em (20\% of the final grade)}}
Attendance in this class is
mandatory, and attendance is defined as full participation for the
entire duration of a class period. Attendance will be taken at the
end of every class period. Partial attendance for a class meeting
will receive no credit. All assignments will be due at the beginning of their 
respective class period and all
quizzes will be given during the first few minutes of their respective
class periods. Failure to submit an assignment and/or failure to take
a quiz will also result in an absence for the day.

In most class meetings, there will be a hands-on activity which we
will complete together as a class.  Should any part of these assignments
not be completed in class, it is your responsibility to complete them
on your own.  In-class lab assignments are due on the Friday of each
week, and are turned in electronically.  No late lab submissions will
be accepted.


\subsubsection*{Programming Assignments {\em (35\% of the final grade)}}
There will be five programming assignments given throughout the year.
You are expected to work independently on these assignments.  The
source code you submit must come from one of three sources:
\begin{itemize}
    \item Your own original code.
    \item Code from class examples and lab assignments.
    \item Code from your textbook.
\end{itemize}

If you do use code from class examples and/or your textbook, you are
required to cite the sections of your code that are not original.
Using source code from any other source will be considered cheating,
and will be dealt with according to the cheating and plagiarism policy
stated later in this syllabus.
(This includes paraphrasing code found on code repository sites such
as github, gitlab, etc.  It also includes using code snippets from
help sites such as stack overflow.  You may use these sites to study,
but you must not ever submit code from these sources as your own!)

\subsubsection*{Term Project {\em (25\% of the final grade)}}
There is a project which is due at the end of the term.  The project
is an open ended project in that you will create a video game of your
chosing.  The constraints of this project are:
\begin{itemize}
    \item The game must be written in C++.
    \item The game must make use of the FLTK graphics library.
    \item The game must consist only of source code form the following
        sources:
        \begin{itemize}
            \item Your original code.
            \item Code from the textbook.
            \item Code from labs and in-class coding examples.
        \end{itemize}
        As is the case with the programming assignments, you must cite
        the source of any code you use in this project. Doing
        otherwise constitutes plagiarism.

    \item At the end of the term, you will make the code public under
        an open source license of your chosing.
\end{itemize}
This project is assigned as of the beginning of the course, and
various milestones will be announced as the semester progresses. 

Your overall letter grade for the course cannot be higher than the letter
grade on this assignment.




% Course Schedule %%%%%%%%%%%%%%%%%%%%%%%%%%%%%%%%%%%%%%%%%%%
\subsection*{Schedule}
This is the tentative schedule for our course.  There may be some
slight modifications to the following according to the needs of the
semester. However, the exam dates are fixed, and will be followed.
All exam dates are shown in boxes, and it is absolutely vital to
your success in this course that you attend the exam days.  The final
exam period is set by the registrar.  Attendance on the date of the
final exam is absolutely mandatory; any student failing to appear on
this date will receive a failing grade for the course.

\subsubsection*{January 2020}
\begin{tabular}{rrrrrrr}
Su & Mo & Tu & We & Th & Fr & Sa\\
   &    &    &  1 &  2 &  3 &  4\\ 
 5 &  6 &  7 &  8 &  9 & 10 & 11\\ 
12 & 13 & 14 & 15 & 16 & 17 & 18\\ 
19 & 20 & 21 & 22 & 23 & 24 & 25\\ 
26 & 27 & 28 & 29 & 30 & 31 &\\
\end{tabular}
\begin{itemize}
\item\textbf{Wed January  8} - Reviewing the Basics
    \begin{itemize}
        \item Review Chapters 1 and 2
        \item Lab Activity: Programming Project P1.2 From Big C++
    \end{itemize}
\item\textbf{Mon January 13} - C++ Design and Thinking
    \begin{itemize}
        \item Review Chapters 2 and 3
        \item Lab Activity: Programming Project P3.3 From Big C++
    \end{itemize}
\item\textbf{Wed January 15} - C++ Design and Thinking
    \begin{itemize}
        \item Review Chapters 4 and 5
        \item Lab Activity: Programming Project P5.9 From Big C++
    \end{itemize}
\item\textbf{Mon January 20} - (MLK Day, no afternoon classes)
\item\textbf{Wed January 22} - C++ Classes and Objects
    \begin{itemize}
        \item Review Chapters 8 and 9
        \item Lab Activity: Programming Project P9.5 From Big C++
        \item Lab Activity: Write a driver program for the battery
            class.
    \end{itemize}
\item\textbf{Mon January 27} - Arrays and Pointer Basics
    \begin{itemize}
        \item Read Chapter 6.1 -- 6.6 and 7.1 -- 7.4
        \item Lab Activity: Programming Project P7.2 from Big C++
    \end{itemize}
\item\textbf{Wed January 29} - Pointer Basics
    \begin{itemize}
        \item Read Chapter 7.5 -- 7.6
        \item Lab Activity: Programming Project P7.1 from Big C++
    \end{itemize}
\end{itemize}
\hrulefill

\subsubsection*{February 2020}
\begin{tabular}{rrrrrrr}
Su & Mo & Tu & We & Th & Fr & Sa\\
   &    &    &    &    &    &  1\\ 
 2 &  3 &  4 &  5 &  6 &  7 &  8\\ 
 9 & 10 & 11 & 12 & 13 & 14 & 15\\ 
16 & 17 & 18 & 19 & 20 & 21 & 22\\ 
23 & 24 & 25 & 26 & 27 & 28 & 29\\ 
\end{tabular}
\begin{itemize}
\item\textbf{Mon February  3} - Objects, Pointers, and Dynamic Allocations
    \begin{itemize}
        \item Read Chapter 7.8 and 9.10
        \item Lab Activity: Programming Problem P9.2 from Big C++
        \item Lab Activity: Implement a driver program using pointers and dynamic
            allocation.
    \end{itemize}
\item\textbf{Wed February  5} - Objects, Pointers, and Dynamic Allocations
    \begin{itemize}
        \item Lab Activity: Programming Problem P9.4 from Big C++
        \item Lab Activity: Fully implement everything using dynamic allocation.
    \end{itemize}
\item\textbf{Mon February 10} - An Introduction to FLTK
    \begin{itemize}
        \item Begin work on Program 1: Pocket Calculator (Due March 2)
        \item Read FLTK 1.3.5 Programming Manual sections 
            ``Introduction to FLTK'' and `` FLTK Basics''
        \item Lab Activity: Create the FLTK Hello World Program
        \item Lab Activity: Create a Makefile template for FLTK
            Applications
    \end{itemize}
\item\textbf{Wed February 12} - An Introduction to FLTK
    \begin{itemize}
        \item Read FLTK 1.3.5 Programming Manual section ``Common
            Widgets and Attributes''
        \item Lab Activity: FTLK Window Layout
    \end{itemize}
\item\textbf{Mon February 17} - Inheritance and Polymorphism
    \begin{itemize}
        \item Read Chapter 10
        \item Lab Activity: P10.7 from Big C++
    \end{itemize}
\item\textbf{Wed February 19} - Inheritance, Polymorphism, and Drawing Shapes
    \begin{itemize}
        \item Read FLTK 1.3.5 Programming Manual section ``Drawing Things in FLTK''
        \item Lab Activity: Drawing Shapes
    \end{itemize}
\item\textbf{Mon February 24} - Events and Shapes
    \begin{itemize}
        \item Read FLTK 1.3.5 Programming Manual section ``Handling Events''
        \item Lab Activity: Click Art
    \end{itemize}
\item\textbf{Wed February 26} - Responding to Events
    \begin{itemize}
        \item Lab Activity: Enhanced Click Art
    \end{itemize}
\end{itemize}
\hrulefill

\subsubsection*{March 2020}
\begin{tabular}{rrrrrrr}
Su & Mo & Tu & We & Th & Fr & Sa\\ \cline{4-4}
 1 &  2 &  3 & \multicolumn{1}{|r|}{4} &  5 &  6 &  7\\ \cline{4-4}
 8 &  9 & 10 & 11 & 12 & 13 & 14\\ 
15 & 16 & 17 & 18 & 19 & 20 & 21\\ 
22 & 23 & 24 & 25 & 26 & 27 & 28\\
29 & 30 & 31 &    &    &    & \\
\end{tabular}
\begin{itemize}
\item\textbf{Mon March  2} - Review
    \begin{itemize}
        \item Program 1: Pocket Calculaor is Due
    \end{itemize}
\item\textbf{Wed March  4} - Midterm Exam
\item\textbf{Mon March  9} - Custom Widgets
    \begin{itemize}
        \item Read FLTK 1.3.5 Programming Manual section ``Adding and Extending Widgets''
        \item Begin work on Program 2: Conway's Game of Life (Due March 30)
        \item Lab Activity: Big Pixel Grid
    \end{itemize}
\item\textbf{Wed March 11} Custom Widgets
    \begin{itemize}
        \item Lab Ativity: Big Pixel Editor
    \end{itemize}
\item\textbf{Mon March 16} Spring Break
\item\textbf{Wed March 18} Spring Break
\item\textbf{Mon March 23} Animations, Games, and Simulations
    \begin{itemize}
        \item Read the FLTK example \texttt{animated.cxx} 
        \item Get animated.cxx running.
        \item Lab Activity: Trajectory
    \end{itemize}
\item\textbf{Wed March 25} Animations, Games, and Simulations
    \begin{itemize}
        \item Lab Activity: Sprite Engine Examples 
    \end{itemize} 
\item\textbf{Mon March 30} Namespaces, Frameworks, and Alpacas
    \begin{itemize}
        \item Program 2: Conway's Game of Life is Due
        \item Begin Work on Program 3: A Pack of Attack Alpacas (Due April 8)
    \end{itemize}
\end{itemize}

\subsubsection*{April 2020}
\begin{tabular}{rrrrrrr}
Su & Mo & Tu & We & Th & Fr & Sa\\
   &    &    &  1 &  2 &  3 &  4\\
 5 &  6 &  7 &  8 &  9 & 10 & 11\\
12 & 13 & 14 & 15 & 16 & 17 & 18\\
19 & 20 & 21 & 22 & 23 & 24 & 25\\ \cline{3-3}
26 & 27 & \multicolumn{1}{|r|}{28} & 29 & 30 &    &\\ \cline{3-3}
\end{tabular}
\begin{itemize}
\item\textbf{Wed April  1} - OOP Backtracking Algorithm
    \begin{itemize}
        \item Read Chapter 11.6 
        \item Lab Activity: Sudoku Solver
    \end{itemize}
\item\textbf{Mon April  6} - OOP Backtracking Algorithm
    \begin{itemize}
        \item Lab Activity: Insanity
    \end{itemize}
\item\textbf{Wed April  8} - Advanced Use of Debuggers
    \begin{itemize}
        \item Program 3: A Pack of Attack Alapacas is Due
        \item Begin Work on Program 4: Pegboard Games (Due April 22)
    \end{itemize}
\item\textbf{Mon April 13} - Machine Code and C
    \begin{itemize}
        \item Read \url{https://learnxinyminutes.com/docs/c/}
        \item Lab Activity: Disassembling a Programming in gdb
    \end{itemize}
\item\textbf{Wed April 15} - The UNIX Memory Model
    \begin{itemize}
        \item Lab Activity: Exploring the Stack
    \end{itemize}
\item\textbf{Mon April 20} - Smashing the Stack for Fun and Profit
    \begin{itemize}
        \item Read: Smashing the Stack for Fun and Profit
        \item Lab Activity: Old School Virus Writing
    \end{itemize}
\item\textbf{Wed April 22} - Game Presentations \& Review
    \begin{itemize}
        \item Program 4: Pegboard Games is Due
        \item Activity: Demonstrate and Play Games
    \end{itemize}
\item\textbf{Tue April 28 3:30PM} - Final Exam
\end{itemize}
\hrulefill


% Fifth Section %%%%%%%%%%%%%%%%%%%%%%%%%%%%%%%%%%%%%%%%%%%

\newpage
\section*{Course Policies}

\subsection*{Late Policy}
No late work will be accepted under any circumstances (except as mercy
and decency may dictate in extremely rare events).

\subsection*{Extra Credit}
No extra credit will be given under any circumstances.

\subsection*{Excused Absences}
In some cases, absences may be excused. These include:
\begin{itemize}
    \item School Sanctioned Events (Sports, Concerts, etc.)
    \item Severe Illness
    \item Family Emergencies
    \item Court Appearance / Jury Duty
\end{itemize}
In the case of a school event, notice must be given at least one week
prior to the absence. The notice must include a signed note from the
faculty or staff member in charge of the event. This note must be
given in physical form, electronic notes will not be accepted.
In the case of illness, a doctor's note is required. Note that
except in extreme circumstances, doctor's appointments do not qualify as a valid reason to miss
a class. Please be respectful of the other students, and schedule
appointments during your free time.

Family emergencies will require some form of proof. Where possible,
you must give advance notice of missing a class. The exception to this
would need to be fairly severe, and hopefully it will not come up.
For court appearances and/or jury duty, you must provide a copy of
your summons. You may redact any details you wish, save for the
actual date and time of your appearance. Court appearances must be
cleared at least one week in advance.

\subsection*{Making Up Excused Absences}
Should you be in a situation in which you receive an excused absence,
this in no way will extend your due dates (excepting extreme
emergencies). You must make up any test at a designated time 
prior to your excused absence. Also, homework or projects must
be submitted prior to the class period in which they are due.

\subsection*{Communication and Extra Help}
You are always welcome at office hours for help with any
questions you may have about the course. For help at other times during the day, stop by or call my office to see if I'm available. You can also contact me by email, but often I can better help you face to face and may respond with a request that
you come to see me. Note that I do not typically respond to email between 5 p.m. and 8 a.m. You may make appointments to see me at other times if your schedule does not permit you to attend my office hours.


\subsection*{Plagiarism and Cheating}
You are expected to do your own work. Never submit the work of others,
never give unauthorized assistance to others, do not use unauthorized
aids during exams, and do not ask for help from other
faculty members without the approval of your professor. Plagiarism and cheating are serious offenses that will not be
tolerated. Explanations regarding these offenses and how they are handled can be found in the MC Student Handbook at\newline
https://www.maryvillecollege.edu/academics/catalog/handbook/section-nine/.\newline
You are expected to read and understand these policies. Offenses on specific assignments, quizzes, or exams will result
in a score of 0 on the relevant assignment, and a letter of censure will be placed in your college file. Repeat offenses will
result in further disciplinary action, including the possibility of failing the course.

\subsection*{Students with Disabilities}
Any student who feels s/he may need learning or physical
accommodation(s) based on the impact of a disability should contact Services for Students with Disabilities to discuss your
specific needs. Please contact 981-8124 to coordinate reasonable accommodations for students with documented
disabilities. The Disability Services office is located in the Learning Center in the basement of Thaw Hall. Undocumented
disabilities will not be accommodated.

\end{document}

\end{document}
